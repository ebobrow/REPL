% Quick start guide
\documentclass[17pt,aspectratio=169]{beamer}
\usetheme{metropolis}

\usepackage{listings}
\usepackage{xcolor}
% \usepackage[normalem]{ulem}

% Title page details
\title{A brief introduction to Autosubst}
% \subtitle{and you should too}
\author{Elliot Bobrow}
\institute{Advised by Stephanie Weirich and Yiyun Liu}
\date[\today]{}

\beamertemplatenavigationsymbolsempty
\setbeamercovered{transparent}

\begin{document}
\begin{frame}
% Print the title page as the first slide
\titlepage
\end{frame}

\begin{frame}{What is Autosubst 2?}
    \begin{itemize}
        \item Tool for proofs about programming languages in Coq
        \item Provides tactics to automatically simplify terms
            \begin{itemize}
                \item Prove equality by simplifying two terms to the same form
            \end{itemize}
    \end{itemize}
\end{frame}

\section{Parallel substitutions}

\begin{frame}{Untyped Lambda Calculus}
    \begin{align*}
        e &::= \only<1>{x}\only<2>{n} \\
        &\,|\quad \lambda \onslide<1>{x.}e \\
        &\,|\quad e_1\, e_2
    \end{align*}
\end{frame}

\begin{frame}{De Bruijn Indices}
    \[
        \lambda x.(\lambda y. y\,(\lambda z.z)) (\lambda y.x\,y)
        \Longrightarrow
        \color{red}\lambda
        \color{black} (\color{blue}\lambda 1\color{black}\,(\color{violet}\lambda 1\color{black}))
        (\color{teal}\lambda \color{red}2\,\color{teal}1\color{black})
    \]
\end{frame}

\begin{frame}{Single Substitutions}
    % \[
    %     e[v_1/x_1]\visible<2->{[v_2/x_2]}\visible<3->{\cdots[v_n/x_n]}
    % \]

    % \[
    %     (\lambda \onslide<1>{x.} a)\,b \longrightarrow a[b/\only<1>{x}\only<2->{1}]\only<3->{[2/2][3/3]\cdots}
    % \]
    \[
        (\lambda a)\,b \longrightarrow a[b/1]\only<2->{[2/2][3/3]\cdots}
    \]
\end{frame}

\begin{frame}{Parallel Substitutions}
    % \[
    %     e[v_1/x_1][v_2/x_2]\cdots[v_n/x_n] \Longrightarrow e[\sigma]
    % \]
    % where
    % \[
    %     \sigma = \only<1>{\{x_1\mapsto v_1,\dots,x_n\mapsto v_n\}}
    %              \only<2>{(v_1,v_2,\dots,v_n)}
    % \]
    \[
        (\lambda a)\,b\longrightarrow a[\sigma]
    \]
    where
    \[
        \sigma = (b,2,3,\dots)
    \]
\end{frame}

\begin{frame}{Primitives}
    \begin{align*}
        \text{id}=(1,2,3,\dots) && \uparrow\,=(2,3,4,\dots)
    \end{align*}
    \vspace{1em}
    \begin{align*}
        e\cdot \sigma &= (e,\sigma(1),\sigma(2),\dots)
        % \visible<2>{e\cdot\text{id} &= (e,1,2,\dots)}
    \end{align*}
\end{frame}

\begin{frame}{Instantiation}
    \begin{align*}
        x[\sigma] &= \sigma(x) \\
        (a\,b)[\sigma] &= (a[\sigma])\,(b[\sigma]) \\
        (\lambda a)[\sigma] &= \lambda (a[1\cdot (\sigma\,\circ\uparrow)])
    \end{align*}
\end{frame}

\begin{frame}{Instantiation Example}
    \only<1>{
    \[
        % \color{gray}\lambda
        \color{black}1\,(\lambda 2\,1)
    \]
    }
    % \only<3->{
    %     \[
    %     \color{red}1\color{black}\longmapsto e
    %     \]
    % }
    \only<2-3>{
        \[
            % \visible<2>{\color{gray}\lambda}
            \color{red}1\color{black}\,(\lambda\color{red}2\color{black}\,1)
            \only<3>{[(e,2,3,\dots)]}
        \]
    }
    \only<4-7>{
        \[
            \only<4>{(\color{red}1\color{black}[(e,2,3,\dots)])}
            \only<5->{e}
            \,
            \only<4-5>{((\lambda \color{red}2\color{black}\,1)[(e,2,3,\dots)])}
            \only<6>{(\lambda (\color{red}2\,\color{black}1)[1\cdot((e,2,3,\dots)\,\circ\uparrow)])}
            % \only<7>{(\lambda (\color{red}2\,\color{black}1)[1\cdot(e[\uparrow],3,4,\dots)])}
            \only<7>{(\lambda (\color{red}2\,\color{black}1)[(1,e[\uparrow],3,4,\dots)])}
        \]
    }
    \only<8>{
        \[
            \color{black}e\,(\lambda e[\uparrow]\,1)
        \]
    }
\end{frame}

% \begin{frame}{Instantiation}
%     \alert{Example}:
%     \begin{align*}
%         \only<1>{(\lambda y.\lambda x.x\,y)\,v &\longrightarrow (\lambda x.x\,y)[v/y] \\}
%         \only<2->{(\lambda\lambda 2\,1)\,v &\longrightarrow (\lambda 2\,1)[(v,2,3,\dots)] \\}
%         \visible<3->{&\longrightarrow \lambda ((2\,1)[
%             \only<3>{1\cdot((v,2,3,\dots)\,\circ\uparrow)}
%             \only<4>{1\cdot(v[\uparrow],3,4,\dots)}
%             \only<5->{(1,v[\uparrow],3,4,\dots)}
%         ]) \\}
%         \visible<6->{&\longrightarrow \lambda ((2[(1,v[\uparrow],3,4,\dots)])\,(1[(1,v[\uparrow],3,4,\dots)])) \\}
%         \visible<7->{&\longrightarrow \lambda (v[\uparrow])\,1}
%     \end{align*}
% \end{frame}

% \section{Scoped syntax}

% \begin{frame}{The Goal}
%     For $n:\mathbb{N}$, \texttt{tm n} is the type of terms with $n$ free variables
% \end{frame}

% \begin{frame}[fragile]{Finite types}
%     \begin{definition}
%         \texttt{fin n}: A type with exactly $n$ members
%         \vspace{0.5em}
%         % \[
%         %     \mathcal{F}_0=\emptyset,\quad\quad\mathcal{F}_{n+1}=(\mathcal{F}_n)_\perp
%         % \]
%         \begin{lstlisting}[basicstyle=\footnotesize\ttfamily,language={[Objective]Caml}]
%         Fixpoint fin (n : nat) : Type :=
%             match n with
%             | 0   => False
%             | S n => option (fin n)
%             end.
%         \end{lstlisting}
%     \end{definition}
% \end{frame}

% \begin{frame}[fragile]{Finite terms}
%     \begin{definition}
%         \texttt{tm n}: A term whose variables are of type \texttt{fin n}
%         \vspace{0.5em}
%         \begin{lstlisting}[basicstyle=\footnotesize\ttfamily,language={[Objective]Caml}]
%         Inductive tm (n : nat) : Type :=
%             | var_tm : fin n -> tm n
%             ...
%         \end{lstlisting}
%     \end{definition}
% \end{frame}

\section{Autosubst in action}

\begin{frame}[fragile]{Syntax specification}
    \begin{figure}[H]
        \centering
        \begin{lstlisting}[language={[Objective]Caml},basicstyle=\footnotesize\ttfamily,keywordstyle=\bfseries]
            tm : Type
            ty : Type

            app : tm -> tm -> tm
            abs : ty -> (tm -> tm) -> tm

            fun : ty -> ty -> ty
            I : ty
        \end{lstlisting}
        \caption{stlc.sig}
    \end{figure}
\end{frame}

\begin{frame}[fragile]{Generated code}
    \begin{figure}[H]
        \centering
        \begin{lstlisting}[language={[Objective]Caml},basicstyle=\footnotesize\ttfamily,keywordstyle=\bfseries]
            Inductive ty : Type :=
              | fun : ty -> ty -> ty
              | I : ty.

            Inductive tm : Type :=
              | var_tm : nat -> tm
              | app : tm -> tm -> tm
              | abs : ty -> tm -> tm.
        \end{lstlisting}
        \caption{stlc.v}
    \end{figure}
\end{frame}

\begin{frame}[fragile]{Example}
    \begin{lstlisting}[mathescape=true,basicstyle=\footnotesize\ttfamily,keywordstyle=\bfseries]
$\sigma$ : nat -> tm
e v : tm
    \end{lstlisting}
    \hrule
    \begin{onlyenv}<1>
    \begin{lstlisting}[mathescape=true,basicstyle=\footnotesize\ttfamily,keywordstyle=\bfseries]
P (app (abs e)[$\sigma$] v)
    \end{lstlisting}
    \end{onlyenv}
    \begin{onlyenv}<2>
    \begin{lstlisting}[mathescape=true,basicstyle=\footnotesize\ttfamily,keywordstyle=\bfseries]
P e[1 $\cdot$ ($\sigma$ $\circ$ $\uparrow$)][v $\cdot$ id]
    \end{lstlisting}
    \end{onlyenv}
    \begin{onlyenv}<3>
    \begin{lstlisting}[mathescape=true,basicstyle=\footnotesize\ttfamily,keywordstyle=\bfseries]
P e[v $\cdot$ $\sigma$]
    \end{lstlisting}
    \end{onlyenv}
\end{frame}

\section{So what have I been working on?}

\begin{frame}{Termination of STLC}
    \begin{theorem}
        For all terms $e$, if $\,\vdash e : \tau$, then there exists a value $v$ such that $e\rightarrow^* v$.
    \end{theorem}
    % \vspace{0.5em}
    % \begin{lstlisting}[basicstyle=\footnotesize\ttfamily,language={[Objective]Caml}]
    % Theorem termination : forall (e : tm 0) T,
    %     has_type e empty_ctxt T ->
    %     exists v, multistep e v.
    % \end{lstlisting}
\end{frame}

\begin{frame}{Proof using named syntax}
    \begin{columns}
        \column{0.5\textwidth}
        \begin{description}
            \item[Lines of code] 866
            \item[Lemma count] 58
            \item[Admit count] 14
        \end{description}
        \column{0.5\textwidth}
        \begin{figure}[H]
            \centering
            \includegraphics[scale=0.2]{./img/sad2.jpeg}
        \end{figure}
    \end{columns}
\end{frame}

\begin{frame}{Proof using Autosubst 2}
    \begin{columns}
        \column{0.5\textwidth}
        \begin{description}
            \item[Lines of code] 313
            \item[Lemma count] 21
            \item[Admit count] 1
        \end{description}
        \column{0.5\textwidth}
        \begin{figure}[H]
            \centering
            \includegraphics[width=\textwidth]{./img/happy.jpg}
        \end{figure}
    \end{columns}
\end{frame}

\begin{frame}{Summary}
    \begin{itemize}
        \item Single substitutions $\rightarrow$ parallel substitutions
        \item Code generation
        \item Autosubst 2 makes proofs easier
    \end{itemize}
\end{frame}

\begin{frame}{Further reading}
    \scriptsize
    Autosubst paper: \url{https://www.ps.uni-saarland.de/Publications/documents/SchaeferEtAl_2015_Autosubst_-Reasoning.pdf}

    Parallel substitutions: \url{http://www.lucacardelli.name/Papers/SRC-054.pdf}

    Autosubst 2 paper: \url{https://www.ps.uni-saarland.de/Publications/documents/StarkEtAl_2018_Autosubst-2_.pdf}

    Scoped terms: \url{https://link.springer.com/chapter/10.1007/11617990_1}

    Logical relations: \url{https://www.cs.uoregon.edu/research/summerschool/summer23/topics.php}
\end{frame}

\end{document}
