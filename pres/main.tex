% Quick start guide
\documentclass{beamer}
\usetheme {default}
% Title page details
\title{Logical Relations in Coq}
% \subtitle{Quick-start guide}
\author{Elliot Bobrow}
\institute{UPenn REPL}
\date{\today}
\beamertemplatenavigationsymbolsempty

\begin{document}
\begin{frame}
% Print the title page as the first slide
\titlepage
\end{frame}

\begin{frame}{Introduction to Logical Relations}
    \begin{theorem}
        Normalization of STLC: For all terms $e$, if $\,\vdash e : \tau$, then there exists a value $v$ s.t. $e\rightarrow^* v$.
    \end{theorem}
    \begin{block}{Proof.}
        By induction on the typing derivation?

        Case T-App $$\frac{\vdash e1 : \tau_2 \rightarrow \tau \quad \vdash e2 : \tau_2}{\vdash e1\,e2 : \tau}$$

        By IH,
        $$
        e1\,e2 \rightarrow^* (\lambda x:\tau_2.e')\,e2 \rightarrow^* e'[v2/x]
        $$
        IH is too weak!
    \end{block}
\end{frame}

\begin{frame}{Introduction to Logical Relations}
    Define a relation $N_\tau$:
    \begin{align*}
        N_{\text{bool}}(e) &\equiv \quad\vdash e:\text{bool} \wedge \exists v . e \rightarrow^* v \\
        N_{\tau_1\rightarrow\tau_2}(e) &\equiv \quad\vdash e: \tau_1\rightarrow\tau_2 \wedge\exists v . e \rightarrow^* v \wedge \forall e'. N_{\tau_1}(e') \Rightarrow N_{\tau_2}(e\, e')
    \end{align*}
\end{frame}
\end{document}
